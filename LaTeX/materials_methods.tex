% !TEX root = ./main.tex
\documentclass[main.tex]{subfiles}
\begin{document}

\subsection{Resources}
The resources for gathering information about the different algorithms and other relevant information, will come from the internet. Wikipedia will be used extensively, since the difficulty of understanding the articles on Wikipedia is not too high, nor too low.
\newline
\\*
The programming language that will be used is Python. Python is an object-oriented programming language that is often compared to pseudocode because of its simplicity and its english-like syntax. Even though Python is one of the slower programming languages, around $10$ times slower than $C++$, the simplicity and ease of understanding outweighs the slowness. 
\newline
\\*
The tests will be performed on an HP laptop running the Linux operating system. Because the tests only will be a comparison of the different algorithms, it does not really matter which computer the tests are performed on, however it is important that all tests are conducted on the same computer. 

\subsection{Methods}
The comparison of algorithms will be split up into two different tests, to test different characteristics of the algorithms. 
\newline
\\*
The first test will be called "Finding largest prime in 60 seconds". Here, the programs will be given 60 seconds to loop the natural numbers starting at $2$. The algorithms will test the number for primality, and if it is shown to be prime, it will be inserted into an array. Once the time runs out, the largest item in the list is recorded as the result for the test. The test will be repeated 3 times to be able to see if there is any variation between the different instances of conducting the test. The algorithms that will be tested here are the following: Brute-Force, Smart Brute-Force, Fermat and Miller-Rabin. The reason Lucas-Lehmer will not be tested here is that it only works on Mersenne numbers, and not all natural numbers are Mersenne numbers. 
\newline
\\*
The second test, called "Biggest Mersenne prime tested in 120 seconds", will test Mersenne primes for primality. Of course, all algorithms should return true, but that is not what is interesting. What will be looked at are the run-times, in order to compare the speed and time complexities of the different algorithms. A test case is deemed a "success" if it finishes within 120 seconds. If it finishes past 120 seconds, or if it does not finish at all withing a reasonable amount of time, i.e. around 5 minutes, it is deemed a "failure". The numbers being tested are the following: 19, 31, 61, 107, 607, 1279, 4423, 9689, 11213, 19937, 23209, 44497, 86243. Note that these numbers are the exponents of the Mersenne prime being tested, and not actually the numbers to test. In this test, all the algorithms will be compared, since all algorithms are able to test mersenne primes. For the probabilistic tests, the constant $k$ will be equal to $10$, an arbitrary number to decide the number of rounds. Why $10$ was chosen, is discussed in the discussion. 
\newline
\\*
In order to conduct these tests, a "driver" function will be used to call the functions and take the time. The code for the different algorithms will be translated from pseudocode into Python by us. The code of the algorithms and driver function will not be written in this document, however they can be found in this essays GitHub repository. The link is in the appendix. 

\end{document}
