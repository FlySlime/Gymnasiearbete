% !TEX root = ./main.tex
\documentclass[main.tex]{subfiles}
\begin{document}

\subsection{Resources}
The resources for gathering information about the different algorithms and other
relevant information, will come from the internet. Wikipedia \cite{wikipedia}
will be used extensively, since the difficulty of understanding the articles on
Wikipedia is not too high, nor too low. \newline

The programming language that will be used is Python \cite{python}. Python is an
object-oriented programming language that is often compared to pseudocode
because of its simplicity and its English-like syntax. Even though Python is
considerably slower \cite{pythonvscpp} than other, more low level, languages, such as C++\cite{cpp},
the simplicity and ease of understanding it outweights the drawback. \newline

The tests will be performed on a $2014$ HP laptop running the Linux operating
system. Since the tests will only be a comparison of the different algorithms,
it does not really matter which computer the tests are performed on. However all
tests have to be conducted on the same computer.

\subsection{Methods}
The comparison of algorithms will be split up into two different tests, to test
different characteristics of the algorithms. \newline

The first test will be called "Finding largest prime in $60$ seconds". Here the
algorithms will be given $60$ seconds to loop through the natural numbers
starting from $2$. The algorithms will test the number for primality, and if it
is shown to be prime, it will be inserted into an array. Once the time runs out,
the largest element in the list is recorded as the result for the test. The test
will be repeated $3$ times to be able to see if there is any variation between
the different instances of conducting the test. The following algorithms will be tested: Brute-Force, Smart
Brute-Force, Fermat and Miller-Rabin. The reason for why Lucas-Lehmer is not
included is that it only works with Mersenne numbers. \newline

The second test, called "Biggest Mersenne prime tested in $120$ seconds", will
test Mersenne primes for primality. All algorithms are expected to return true,
which is why the runtimes are to be noted instead. This is in order to compare
the speed and time complexities of the different algorithms. A test case is
deemed a ``success'' if it finishes within $120$ seconds. If it exceeds $120$
seconds, or if it still running after five minutes, it is deemed a ``failure''.
A test case that succeeds is coloured green, and red for a failure. The numbers
used for the test are the following: $19$, $31$, $61$, $107$, $607$, $1279$,
$4423$, $9689$, $11213$, $19937$, $23209$, $44497$ and $86243$. Note that these
numbers are the exponents of the Mersenne prime being tested, and not
the actual numbers being tested. \\

All the algorithms will be compared, since all algorithms are able to test
mersenne primes. For the probabilistic algorithms, the constant $k$ will be
equal to $10$, an arbitrary number to decide the number of rounds. \newline

In order to conduct these tests, a ``driver'' function will be used to call the
algorithms and start a timer. The code for the different algorithms will be
translated from pseudocode into Python and can be found at the study's GitHub
page \cite{github}. \\

Driver function for test 1:

\begin{python}
  Insert code here
\end{python}

\vspace{5mm}

Driver function for test 2:

\begin{python}
  Insert code here
\end{python}


\end{document}
