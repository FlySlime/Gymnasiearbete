% !TEX root = ./main.tex
\documentclass[main.tex]{subfiles}
\begin{document}

Prime numbers are numbers that are only divisible by 1 and itself. Apart from
being beautiful mathematically in a number of instances, prime numbers have some
specific but important uses in everyday life. A common example is in
RSA-encryption, where a product of two extremely large prime numbers is
explioted for its toughness to factor, in order to encrypt messages that can
only be decrypted by someone who has the key, i.e. one of the factors.

With the rise of more and more powerful computers, these factors are becoming
easier to find, thus breaking the RSA-encryption. Therefore it is essential to
find larger primes, to avoid breaking the RSA-encryption.

Finding larger primes is however far from trivial. Since the largest primes
known to man today has 24 862 048 digits, using easy methods we learn in school
will not work, not even for the fastest computer on earth. Therefore, newer and
efficient algorithms are of upmost importance.

This study will focus on such algorithms, and compare them in their
effectiveness of finding prime numbers. The following questions arise:

\begin{enumerate}
\item What are some algorithms for primality testing?
\item Which algorithm is the most efficient and why?
\item How far away from the current largest prime can we get?
\end{enumerate}

\end{document}
