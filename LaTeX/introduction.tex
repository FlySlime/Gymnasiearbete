% !TEX root = ./main.tex
\documentclass[main.tex]{subfiles}
\begin{document}

Prime numbers are a beautiful mathematical concept. Out of all other numbers,
primes have had an invaluable importance to our lives. The most common example
being RSA-encryption, where a product of two \emph{extremely} large primes are
exploited for its toughness to factor. In order to decrypt the message, one
would need to have one of the factors. \\

With the rise of more and more powerful computers, these factors are becoming
easier to find, thus breaking the weaker RSA-encryption. Therefore it
is essential to find larger primes. \\

Finding larger primes is however far from trivial. Since the largest prime known
to man has $24,862,048$ digits \cite{prime:largest_digits}. Using a method
that \emph{Checks all numbers below n} is simply not efficient enough for large
primes. \\

The study will focus on efficient algorithms for finding primes, and compare
them. The following questions arise:

\begin{enumerate}
\item What are some algorithms for primality testing?
\item Which algorithm is the most efficient and why?
\item How far away from the current largest prime can we get?
\end{enumerate}

\end{document}
