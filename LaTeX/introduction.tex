% !TEX root = ./main.tex
\documentclass[main.tex]{subfiles}
\begin{document}

Prime numbers are mathematically beautiful. Primes have many uses in everyday
lift, even though we might not think about it that ofte. The most common example
is in RSA-encryption \cite{rsa}, where a product of two \emph{extremely} large primes are
exploited for its toughness to factor. However, if one of the factors in known,
it is easy to find the other. If no factors are known, you would need to factor
the prime youself which is practically impossible.\\

With the rise of more and more powerful computers, these factors are becoming
easier to find, thus breaking the weaker RSA-encryption. Therefore it
is essential to find larger primes. \\

Finding larger primes is however far from trivial. Since the largest prime known
to man has $24,862,048$ digits \cite{prime:largest_digits}, using a method
that \emph{Checks all numbers below n} is simply not fast enough for large
primes. \\

This study will focus on five simple algorithms for finding primes, and compare
them. The following questions arise:

\begin{enumerate}
\item What are some algorithms for primality testing?
\item Which algorithm is the most efficient and why?
\item How far away from the current largest prime can we get?
\end{enumerate}

\end{document}
