% !TEX root = ./main.tex
\documentclass[main.tex]{subfiles}
\begin{document}

Prime numbers have many uses in everyday life, although not noticeable. The most
common example is RSA-encryption \cite{rsa}, where a product of two
\emph{extremely} large primes are used as a key for its toughness to factor. If
one of the factors were to be known, obtaining the other factor would be no
challenge. If no factors are known, then the only solution is to factor the
prime for hand, which is practically impossible. \\

In the age of computers, keys that used to be considered impossible are now
easily cracked. With the rise of more and more powerful computers, these factors
are becoming easier to find, thus breaking the weaker RSA-encryption. Therefore it is essential to find larger primes. \\

Finding larger primes is however far from trivial. Since the largest prime known
to man has $24,862,048$ digits \cite{prime:largest_digits}, using a method that
\emph{Checks all numbers below n} is simply not fast enough for large
primes. \\

This study will focus on five algorithms for finding primes, and compare them.
The following questions arise:

\begin{enumerate}
\item What are some algorithms for primality testing?
\item Which algorithm is the most efficient and why?
\item How far away from the current largest prime can we get?
\end{enumerate}

\end{document}
