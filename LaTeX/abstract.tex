% !TEX root = ./main.tex
\documentclass[main.tex]{subfiles}
\begin{document}

Prime numbers are a fundamental part of number theory. Primes have many special
properties that can be exploited. Perhaps, the most important use is in
RSA-encryption which relies upon the toughness to factor a number consiting of
the product of two large primes. The rise of more powerful computers has
increased the threshold for how large numbers a computer can handle. Therefore
it is of a computer scientist's interest to find and verify larger and larger
primes. In ordet to accomplish this, efficient algorithms have to be developed. \\

This study investigates the efficiency of five simple primality testing
algorithms:
\begin{itemize}
  \item Brute-force
  \item Smart brute-force
  \item Lucas-Lehmer
  \item Miller-Rabin
  \item Fermat primality test
\end{itemize}

In the first test, the algorithms were given the task of finding the largest
prime number within $60$ seconds. The test was performed by looping through the
natural numbers starting from $2$, and calling the primeTest function for every
number in the loop, as long as the time limit was not exceeded. The largest
number tested until the program terminated was the result of the test case. This
test did not inclue Lucas-Lehmer, as it only works for mersenne numbers. \\

In the second test, the algorithms were given mersenne primes of different sizes
and returned the time it took for them to finish verifying that it was indeed a
prime. However, if the time exceeded $120$ seconds, the test case would be
deemed as a ``failure'' and immediately end the program. \\

In the first test, the Fermat primality testing algorithm could find the largest
primes within the time limit, thus concluding that it was the fastest algorithm
out of the ones tested. In the second test, however, Lucas-Lehmer was the
fastest out of all algorithms, thereby concluding that it is the fastest
algorithm of all five. 

\end{document}
