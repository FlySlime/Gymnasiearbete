% !TEX root = ./main.tex
\documentclass[main.tex]{subfiles}
\begin{document}

Prime numbers are a fundamental part of number theory. Primes have many special
properties that can be exploited. One of the most common applications being
RSA-encryption, which uses a product of two large primes as a key. The rise of
more powerful computers has led to weaker keys being cracked. Therefore it is in
a Computer Scientists' interest to find larger primes. In order to accomplish
this, efficient algorithms have to be developed. This study investigates the
efficiency of five simple primality testing algorithms:
\begin{multicols}{3}
  \begin{itemize}
  \item Brute-force
  \item Smart Brute-force
  \item Lucas-Lehmer
  \item Fermat Primality Test
  \item Miller-Rabin
  \end{itemize}
\end{multicols}

In the first test, the algorithms were given the task of finding the largest
prime number within $60$ seconds. The test is performed by looping through the
natural numbers starting from $2$. The first test could not include
Lucas-Lehmer, as it only works for mersenne numbers. In the first test it was
determined that the Fermat primality test found the largest prime, followed by Lucas-Lehmer. \\

In the second test, the algorithms were given mersenne primes of different sizes
and returned the time it took for them to finish verifying that it was a prime.
However, if the time exceeded $120$ seconds, the test case would be deemed as a
``failure'' and immediately end the program. These tests concluded that
Lucas-Lehmer was the fastest algorithm to verify mersenne primes. \\

The purpose of this study is not to find newer primes, but rather to demonstrate
the correlation between the Big O notation and runtimes, as-well as giving an
introduction to primality testing algorithms. \\

\end{document}
