% !TEX root = ./main.tex
\documentclass[main.tex]{subfiles}
\begin{document}

Prime numbers are a fundamental part of number theory. They have many important
properties and it is therefore of some number theorists interests to find larger
and larger primes. In order to accomplish this, efficient algorithms have to be developed. This study investigates the efficiency of five
common primality testing algorithms: brute-force, smart brute-force, fermat,
miller-rabin and lucas-lehmer. Two different tests were performed.

In the first test, the algorithms were given the task of finding the largest prime number
within 60 seconds by looping the natural numbers from 1, returning their largest
confirmed test case once the timer ran out. This test did not include
Lucas-Lehmer, as it only works for mersenne numbers.

In the second test, the
algorithms were given mersenne primes of different sizes and returned the time
it took for them to finish verifying that it was indeed a prime. However, if the
time exceeded 120 seconds, the test case would be deemed as a 'failure' and no
further test cases would be ran. These tests showed that the Fermat primality
test was the fastest and most efficient in both tests.

This study has little to no use in actually finding new primes, as much more
efficient algorithms are currently used for this purpose. However, this study
can be used to demonstrate the correlation between the big O notation and
runtimes, aswell as giving an introduction to simple primality testing
algorithms. 

\end{document}