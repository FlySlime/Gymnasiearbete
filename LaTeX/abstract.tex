% !TEX root = ./main.tex
\documentclass[main.tex]{subfiles}
\begin{document}

Prime numbers are a fundamental part of number theory. Large primes contain
valuable properties that can be exploited, therefore it is in the interest of
number theorists to find these primes. In order to accomplish this, efficient
algorithms have to be developed. This study investigates the efficiency of five
common primality testing algorithms: Brute-force, Smart Brute-force,
Lucas-Lehmer, Fermat and Miller-Rabin. Two different tests were performed to
compare the algorithms. \\

In the first test, the algorithms were given the task of finding the largest
prime number within $60$ seconds. The test is performed by looping through the
natural numbers starting from $2$. The first test could not include
Lucas-Lehmer, as it only works for mersenne numbers. \\

In the second test, the algorithms were given mersenne primes of different sizes
and returned the time it took for them to finish verifying that it was indeed a
prime. However, if the time exceeded $120$ seconds, the test case would be
deemed as a ``failure'' and immediately end the program. These tests concluded
that Lucas-Lehmer was the fastest algorithm to find primes. The other
alternative being the Fermat primality test, with the drawback of being probabilistic. \\

The purpose of this study is not to find newer primes, but rather to demonstrate
the correlation between the big O notation and runtimes, as-well as giving an
introduction to primality testing algorithms. \\

\end{document}
