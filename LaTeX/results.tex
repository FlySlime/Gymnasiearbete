% !TEX root = ./main.tex
\documentclass[main.tex]{subfiles}
\begin{document}

\subsection{Primality Testing}
There are many different ways to test if a natural number, $n$, is prime or composite. The most trivial way is to check every number from $2$ to $n - 1$ and see whether or not they divide $n$ evenly. If no natural number is a factor of $n$, the program will return that $n$ is a prime. The described algorithm is usually called \textit{Brute-force}. 

\begin{python}
def primeTest(n):
    for i in range(2, p - 1):
        if n % i == 0:
            return "Composite"
    return "Prime"
\end{python}

The run time of this algorithm was XXX seconds. This method runs in $O(n)$ time, which is relatively slow.  $O(n)$ lets us check numbers less than or equal to approximately XXX. Compared to other methods, this turned out to be the slowest.

This method can be improved quite a bit though after realizing two important things:
1. We do not need to check all integers up to p, it is sufficient only to test numbers up to sqrt(p)
2. We do not need to check any even numbers, except for the number 2. This is because all even numbers have 2 as a factor, and therefore 2 would already be a known factor of p, thus concluding that p is composite.

Rewriting the program:

\begin{python}
def primeTest(n):
    if n % 2 == 0 or n % 3 == 0:
        return "Composite"
    for i in range(5, sqrt(n)):
        if n % i == 0:
            return "Composite"
    return "Prime"
\end{python}

This resulted in the run time of XXX seconds. The time complexity. 

The 

\subsection{Generating Largest Prime}

\end{document}