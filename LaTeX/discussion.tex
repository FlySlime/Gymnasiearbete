% !TEX root = ./main.tex
\documentclass[main.tex]{subfiles}
\begin{document}
% Answer questions
% Why k = 10?
% What could our tests imply?
% Importance of results.

As showed in both tests, the Brute-force was the slowest of all algorithms tested, and
could only handle numbers up to the magnitude of $10^{9}$. This makes sense
considering the time complexity, $O(n)$. Since a normal computer can do a little
more than $10^{8}$ operations per second, a result of 280 seconds is expected.
For the algorithm to finish $2^{61}-1$, it would take around $10^{10}$ seconds,
which is more than 300 years. This demonstrates the fact that efficient
algorithms must be found. As $O(n)$ grows linearly, it does not take much for
the run time to exceed our lifetimes. \\

The smart brute-force, with its time complexity of $O(sqrt(n))$, was
considerably faster and completed $p=61$ in 100 seconds, rather than 300 years
as in the case of the normal brute-force. However, this does not give the
algorithm much hope, because $p=107$ exceeded the time limit of 120 seconds. \\

On the other hand, Lucas-Lehmer could handle $2^{44497}$, which is approximately
equal to $10^{13394}$. This is
undoubtedly better than both brute-forces, and shows how important it is to take
advantage of the properties of Mersenne primes in order to make a much more
efficient deterministic algorithm. \\

Even though the results showed that these relatively simple algorithms could
handle numbers on the order of magnitude of $10^{13,394}$, this is far from the
largest prime known to man. The reason behind this is the limitations of this
essay. In order to find larger prime numbers than those tested in this essay,
even more efficient algorithms must be used, as well as more powerful computers.
However, these algorithms are incredibly hard to even understand, and are even
harder to code. \\

Overall, the algorithms tested correlate well to their respective big O
notation. However, there is still a non-negligible variance. This is especially
shown in test 4.1 where all of the different sub-tests for each algorithm
yielded different numbers. The explanation for this phenomena is that the
performance of a computer is not constant. The change of CPU temperature, clock
speed and prioritisation of processes within the operating system, all lead to
change in performance. These factors are all beyond the control of the user. In
hindsight, it would be wise to have several sub-tests in 4.2 as-well, since it
was clearly shown that performance fluctuates over time, and that the tests done
in 4.2 therefore are not reliable. This is something that should be done if this
study were to ever be re-done. \\

The most efficient algorithm which was tested by this study was the Lucas-Lehmer
algorithm, as shown in 4.2. It could handle $2^{44,497}-1$. This is however
nowhere near the current largest prime, $2^{82,589,933}-1$. There are several
different reasons for this. The current largest prime was not even found using
the Fermat primality test. The algorithm used to find this large number is
a more optimized version of the Lucas-Lehmer. This version is extremely complicated and is far beyond the scope of this study, since it
requires a ph.D. to even understand. Even if the Fermat primality test were used
to find this large prime, we would not be able to recreate it. The laptop used
in this study is nothing compared to the GIMPS coalition of many computers, each
one of them more powerful than this laptop. \\

A future test regarding this subject could be improved by testing different
values for the constant $k$ and exploring how these different values would
affect the run times of the probabilistic algorithms. Another area of
improvement would be to increase the number of test cases in both 4.1 and 4.2 in
order to 1. draw conclusions regarding the variance and 2. draw conclusions
regarding the accuracy of the run times in test 4.2. Only having one test case
per algorithm and $p$ is like conducting a survey with one test person, it is
not accurate at all. \\

The constant $k$ was chosen to be $10$, which is a relatively arbitrary number.
Research on the internet did not yield a definite answer as to what value $k$
should have. Some algorithms had $k = 3$ while others had $k = 10$. The run
times of the probabilistic algorithms are proportional to this $k$ constant. Therefore,
testing different values of $k$ for each algorithm would be essential to
determine the actual efficiency of the probabilistic algorithms. Another aspect
of having an arbitrary $k$ is that the accuracy of the tests might be impeded.
No control was made to make sure that the algorithms actually returned a true
boolean value. This would also be a great area of improvement, as this fact
basically deems these tests as useless. \\ 

\end{document}
