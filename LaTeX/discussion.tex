% !TEX root = ./main.tex
\documentclass[main.tex]{subfiles}
\begin{document}

Both tests showed that the Brute-force was the slowest of all algorithms
tested, and could only handle numbers up to the magnitude of $10^{9}$. This
makes sense considering the time complexity, $\mathcal{O}(N)$. Since a modern
computer can do a little more than $10^{8}$ operations per second, a result of
$280$ seconds is very reasonable. For the algorithm to verify $2^{61}-1$, it would take
around $10^{10}$ seconds, which is more than $300$ years. This demonstrates the
fact that efficient algorithms have to be used in order to find larger primes. As $\mathcal{O}(N)$ grows
linearly, it does not take much for the runtime to exceed our lifetimes. \\

The Smart Brute-force, with its time complexity of $\mathcal{O}(\sqrt{N})$, was
considerably faster and completed $p=61$ in $100$ seconds, rather than $300$
years as in the case of the normal brute-force. However, that still doesn't
increase the algorithm's performance by much, since it exceeds the time-limit on
the next test, $p=107$. \\

On the other hand, Lucas-Lehmer could handle $2^{44,497}$, which is approximately
equal to $10^{13,394}$. Compared to the Brute-force algorithms, this is a
substantial step for the deterministic algorithms. Lucas-Lehmer demonstrates
how efficient an algorithm can become if it exploits the properties of Mersenne primes. \\

According to the tests, the Fermat primality test clocks in at a second place,
not far behind Lucas-Lehmer. The interpretation of this result must however be
taken with a pinch of salt. Only one arbitrary value of $k$ was tested, and it
would be interesting to see how this value changes the runtime of the algorithm.
Another method flaw is that no verification was done to ensure that the
algorithm returned a correct boolean value. Since this algorithm is
probabilistic, it is essential for such a test to also verify that the algorithm
does not give false positives or negatives. This fact actually deems the tests
useless when it comes to the probabilistic algorithms.\\

As for the Miller-Rabin algorithm, the same flaws still apply and therefore it
is not possible to compare it to the other algorithms. One would think that it
is at least possible to compare Miller-Rabin to Fermat, as their time
complexities both are proportional to $k$ and that the value of $k$ still would
not affect their relative performance. This is partly true and partly false. For
test 1, this will not work since there still is not a validity test. For test 2,
however, this is true. Miller-Rabin is therefore concluded to be slower than Fermat.

Overall, the algorithms tested correlate well to their respective Big O
notation. However, there is still a non-negligible variance. This is especially
shown in test (\ref{test1}) where all of the different sub-tests for each
algorithm yielded different numbers. The explanation for this phenomena is that
the performance of a computer is not constant. The change of CPU temperature,
clock speed and prioritisation of processes within the operating system, all
lead to change in performance. These factors are all beyond the control of the
user. In hindsight, it would be wise to have several sub-tests in (\ref{test2})
as-well, since it was clearly shown that performance fluctuates over time, and
that the tests done in (\ref{test2}) therefore are not reliable. This is
something that should be done if this study were to ever be re-done. \\

Even though the deterministic algorithms managed to verify numbers up to
$10^{13,394}$, it is not even close to the largest prime known to man,
$2^{82,589,933} - 1$ \cite{prime:largest_digits}. There are many factors as to
why the results do not come near the largest prime. One such factor is that the
algorithms explained in this essay are relatively simple to implement and
understand. The
algorithms used in finding the largest primes are \emph{extremely} complicated
to understand and especially implement. Other factors can be time and computers, which are huge limits for this study. \\

The most efficient algorithm which was tested by this study was the Lucas-Lehmer
algorithm, as shown in (\ref{test2}). It could handle $2^{44,497}-1$. This is
however nowhere near the current largest prime, $2^{82,589,933}-1$. There are
several different reasons for this. The current largest prime was not even found
using the Fermat primality test. The algorithm used to find this large number is
a more optimised version of the Lucas-Lehmer. This version is extremely
complicated and is far beyond the scope of this study, since it requires a ph.D.
to even understand. Even if the Fermat primality test were used to find this
large prime, we would not be able to recreate it. The laptop used in this study
is nothing compared to the GIMPS coalition of many computers, each
one of them more powerful than this laptop. \\

A future test regarding this subject could be improved by testing different
values for the constant $k$ and exploring how these different values would
affect the runtimes of the probabilistic algorithms. Another area of improvement
would be to increase the number of test cases in both (\ref{test1}) and
(\ref{test2}) in order to 1. draw conclusions regarding the variance and 2. draw
conclusions regarding the accuracy of the runtimes in test (\ref{test2}). Only
having one test case per algorithm and $p$ is like conducting a survey with one
test person, it is not accurate at all. \\

The constant $k$ was chosen to be $10$, which is a relatively arbitrary number.
Research on the internet did not yield a definite answer as to what value $k$
should have. Some algorithms had $k = 3$ while others had $k = 10$. The run
times of the probabilistic algorithms are proportional to this $k$ constant.
Therefore, testing different values of $k$ for each algorithm would be essential
to determine the actual efficiency of the probabilistic algorithms. Another
aspect of having an arbitrary $k$ is that the accuracy of the tests might be
impeded. No control was made to make sure that the algorithms actually returned
a true boolean value. This would also be a great area of improvement, as this
fact basically deems these tests as useless. \\

\end{document}
