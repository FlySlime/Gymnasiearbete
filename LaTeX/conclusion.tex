% !TEX root = ./main.tex
\documentclass[main.tex]{subfiles}
\begin{document}

Overall, the fastest algorithm was Lucas-Lehmer, closely followed by the Fermat
primality test. The reason Lucas-Lehmer was the best is because it exploits the
properties of Mersenne primes. However, it is impossible to draw a good
conclusion regarding which algorithm is the most efficient. This is because of
the method flaws, such as not testing different values of $k$ and having too few
test cases in both tests. This would need to be improved
should the same tests be conducted again. \\

It was concluded that we could not get anywhere close to the current largest
prime because this study did not include more efficient tests, as well as the
limitations of the computer. \\

Even though the tests conducted can be deemed as failures because of the method
flaws, this study still accomplishes its goal, namely to give the authors an
introduction to applied programming and number theory. This study should not be
used for scientists to expand their knowledge, but rather for young people as an
introduction to aforementioned subjects.

\end{document}
