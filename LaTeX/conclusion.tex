% !TEX root = ./main.tex
\documentclass[main.tex]{subfiles}
\begin{document}

To conclude, the fastest algorithm was Lucas-Lehmer, closely followed by the
Fermat primality test. Lucas-Lehmer averaged better since it specialises in
Mersenne primes, and therefore larger numbers. It is hard to determine which
algorithm was most efficient because of this reason. The algorithms specialise
in something different, i.e. the Fermat primality test was fast, with the
drawback of being probabilistic. Fermat and Miller-Rabin also depend on the
different values of $k$, giving an error in the methods; \emph{should one
  sacrifice accuracy for speed}? \\

The study could not get near the largest prime number. The three factors for
this being:

\begin{enumerate}
\item The \textbf{algorithms} chosen were not efficient enough, but rather to be
  simple
\item \textbf{Computational} power was lacking
\item There wasn't enough \textbf{time}
\end{enumerate}

The study accomplished its goal, to give an introduction to applied programming
and number theory. It is not intended for scientists, mathematicians or experts
in this field, instead it it should be used to introduce students or interested
individuals to the aforementioned subjects.

\end{document}
