% !TEX root = ./main.tex
\documentclass[main.tex]{subfiles}
\begin{document}

To conclude, the fastest algorithm tested was Lucas-Lehmer. Because of the
method flaws named in the discussion, it is not possible to compare the
deterministic algorithms with the probabilistic algorithms. Out of the
probabilistic, the Fermat primality test performed better. This raises a
rethorical question: \emph{should one sacrifice accuracy for speed}?

The study could not get near the largest prime number. The three factors for
this being:

\begin{enumerate}
\item The \textbf{algorithms} chosen were not efficient enough, but rather to be
  simple
\item \textbf{Computational} power was lacking
\item There was not enough \textbf{time}
\end{enumerate}

The study accomplished its goal, to give an introduction to applied programming
and number theory. It is not intended for scientists, mathematicians or experts
in this field, instead it should be used to introduce students or interested
individuals to the aforementioned subjects.

\end{document}
