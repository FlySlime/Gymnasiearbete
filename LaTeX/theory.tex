% !TEX root = ./main.tex
\documentclass[main.tex]{subfiles}
\begin{document}

\subsection{What Is a Prime?}
Prime numbers are defined as \textit{positive integers} which only have the factors 1 and itself. Thus $4$ is not a prime since $4 = 2 * 2$. On the other hand $5$ is a prime since the only divisors of $5$ is $1$ and $5$. An exception to this definition is $1$, since it's the first natural number.

If a number, $n$, is not prime, it is refered to as a \textit{composite number}.

\subsection{The Fundamental Theory of Arithmetic}
The Fundamental Theory of Arithmetic \cite{theorem:arithmetic} states that all integers greater than $1$ is either a prime, or can be expressed as a product of primes in a unique way. This means that all natural numbers, except for $1$, has its own factorization containing only primes, unless it is a prime itself.
\newline
\\*
Important to know is that there is an infinite amount of primes. The proof is a quite easy by contradiction, but nonetheless beautiful:

\begin{mdframed}
    Assume that there is a finite amount of primes and make a list of them:

    $p_1, p_2, p_3, p_4, p_5, ...$ 
    \newline
    \\*
    Let the constant $Q$ be the product of all the primes in the list and add 1:

    $Q = p_1 * p_2 * p_3 * ... + 1$
    \newline
    \\*
    According to the fundamental theorem of arithmetic, $Q$ must be a prime since none of the primes in the list divide $Q$ evenly because of the $1$; therefore making the list incomplete and proving that you cannot make a finite list of all primes. 
\end{mdframed}

\subsection{The Prime Number Theorem}
The Prime Number Theorem \cite{theorem:prime_num} describes approximately how many primes there are less than or equal to a given number. The function $\pi(N) \sim \frac{N}{ln(N)}$ gives the expected amount of primes below a certain $N$. Graphing this function shows that primes become less common for greater $N$.

\begin{figure}
    \begin{center}
        \begin{tikzpicture}
            \begin{axis}[
                axis lines = left,
                xlabel = $N$,
                ylabel = {},
            ]
            \addplot [
                domain=1:100000, 
                samples=100, 
                color=red,
            ]
            {x/ln(x)};
            \addlegendentry{$\frac{N}{ln(N)}$}
            
            \addplot [
                domain=1:100000, 
                samples=100, 
                color=blue,
                ]
                {x};
            \addlegendentry{$N$}
            \end{axis}
        \end{tikzpicture}
    \end{center}
\caption{The graph of $\pi(N)$ and $N$ from $0$ to $10^{5}$ letting us compare the relationship between the number $N$ and the approximate amount of primes below it.}
\end{figure}

This proves that primes do not show up linearaly, meaning a computer that is twice as powerful will \textit{not} produce twice as many primes. Instead, the most important and crucial part of generating and verifying primes is the optimization of the \textit{algorithms}.

\subsection{Time Complexity}

Time complexity \cite{theorem:time_comp} is a concept within computer science, which describes the approximate time for a program to complete. The study will make heavy use of the Big O Notation \cite{theorem:big_O}, which notates how the run time increases as the input size increases. For example, $O(N)$ will grow linearly with the input size. Increasing the input size by a factor of 10, will also increase the run time by a factor of 10, as such $O(10N)$. On the other hand, $O(log(n))$ grows logarithmically, which is far more efficient for bigger input sizes, as $O(log(N))$ is strictly smaller than $N$ for large enough values. It is important to note that when using logarithms in these instances, the base is irrelevant. The proof as to why the base is irrelevant will not be provided by this study.
\newline
\\*
The Big O Notation will be used to determine whether an algorithm with a large number, $n$, will succeed or run for a \textit{very long time}\footnote{Some programs will not finish until the sun explodes, which is quite impractical.}. Therefore it is important to write as efficient algorithms as possible, considering the fact that the largest known prime has $24,862,048$ digits \cite{prime:largest_digits}. 

\subsubsection{Examples of Big O Notation}

A program that runs in $O(N)$ would be a function that inputs an integer $N$ and outputs every number up to $N$:
\begin{python}
def linearTime(N):
    for number in range(N):
        print(number)
\end{python}

Since the program runs in $O(N)$ time, increasing the input by a factor of $10$ would also increase the number of operations done by a factor of $10$.
\newline
\\*
The second example is a program that runs in quadratic time, i.e. $O(N^{2})$. Compared to the first example, this program will print every number up to $N$, $N$ times.

\begin{python}
    def quadraticTime(N):
        for number in range(N):
            for number in range(N):
                print(number)
\end{python}

\vspace{10mm}

The third program describes $O(\sqrt{n})$. It is a simple program that sums all numbers up to $\sqrt{N}$.

\begin{python}
    from math

    def sqrtTime(N):
        sum = 0
        for number in range(math.isqrt(N)):
            sum += number
        print(sum)
\end{python}

\vspace{10mm}

The fourth and last example will be about $O(log (n))$. The algorithm that is used in this example is called \textit{Binary Search} \cite{binary_search}. It's an algorithm used for quickly guessing a number between, for example, $1$ and $100$. Usually two players are required to play this guessing game, but with a computer the user will give an input that the computer will try to guess. The output will be the amount of "guesses" the algorithm perfomed.

\begin{python}
    def binarySearch():
        begin = int(input("The number will be between _ and _\n"))
        end = int(input())
        value = int(input("What value will you input?\n"))
        guesses = 0

        while True:
            guesses += 1
            mid = int((begin + end) / 2)

            if mid > value:
                end = mid - 1
            elif mid < value:
                begin = mid + 1
            else:
                break

    print(guesses)

\end{python}

%3. explain mersenne primes
%4. pi(n) amount fo priems
%4. explain (something more maybe)


\end{document}