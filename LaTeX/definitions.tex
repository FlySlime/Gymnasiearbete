\documentclass[main.tex]{subfiles}
\begin{document}

In order to fully understand this essay, some key concepts have to be adressed and understood.

\subsection{What is a prime?}
Prime numbers are defined as \textit{positive integers} which only have the factors 1 and itself. Thus $4$ is not a prime since $4 = 2 * 2$. On the other hand $5$ is a prime since the only divisors of $5$ is $1$ and $5$. 

If a number, $n$, is not prime, it is refered to as a \textit{composite number}.

\subsection{The Fundamental Theory of Arithmetic}
The Fundamental Theory of Arithmetic \cite{theorem:arithmetic} states that all integers greater than $1$ is either a prime, or can be expressed as a product of primes in a unique way. This means that all natural numbers, except for $1$, has its own factorization containing only primes, unless it is a prime itself.
\newline
\\*
Important to know is that there is an infinite amount of primes. The proof is a quite easy by contradiction, but nonetheless beautiful:

\begin{mdframed}
    Assume that there is a finite amount of primes and make a list of them:

    $p_1, p_2, p_3, p_4, p_5, ...$ 
    \newline
    \\*
    Let the constant $Q$ be the product of all the primes in the list and add 1:

    $Q = p_1 * p_2 * p_3, ... + 1$
    \newline
    \\*
    According to the fundamental theorem of arithmetic, $Q$ must be a prime since none of the primes in the list divide $Q$ evenly because of the $1$; therefore making the list incomplete and proving that you cannot make a finite list of all primes. 
\end{mdframed}

\subsection{The Prime Number Theorem}
The Prime Number Theorem \cite{theorem:prime_num} describes approximately how many primes there are less than or equal to a given number. The function $\pi(N) \sim \frac{N}{ln(N)}$ gives the expected amount of primes below a certain $N$. Graphing this function shows that primes become less common for greater $N$.

\begin{center}
\begin{tikzpicture}
\begin{axis}[
    axis lines = left,
    xlabel = $N$,
    ylabel = {},
]
\addplot [
    domain=1:100000, 
    samples=100, 
    color=red,
]
{x/ln(x)};
\addlegendentry{$\frac{N}{ln(N)}$}

\addplot [
    domain=1:100000, 
    samples=100, 
    color=blue,
    ]
    {x};
\addlegendentry{$N$}

\end{axis}
\end{tikzpicture}
\end{center}

This proves that primes do not show up linearaly, meaning a computer that is twice as powerful will \textit{not} produce twice as many primes. Instead, the most important and crucial part of generating and verifying primes are the \textit{algorithms}.

\subsection{Time Complexity}

Time complexity \cite{theorem:time_comp} is a concept within computer science, which describes the approximate time for a program to complete. The study will make heavy use of the Big O Notation \cite{theorem:big_O}, which notates how the run time increases as the input size increases. For example, $O(N)$ will grow linearly with the input size. Increasing the input size by a factor of 10, will also increase the run time by a factor of 10, as such $O(10N)$. On the other hand, $O(log(n))$ grows logarithmically, which is far more efficient for bigger input sizes. 

This is very important when it comes to testing large numbers, because it makes it easier to determine whether the program will succeed or run for a \textit{very long time}\footnote{Some programs will not finish until the sun explodes, which is quite impractical.}. It is therefore important to write as efficient algorithms as possible, considering the fact that the largest known prime has $24,862,048$ digits \cite{prime:largest_digits}. 

%3. explain mersenne primes
%4. pi(n) amount fo priems
%4. explain (something more maybe)

\end{document}